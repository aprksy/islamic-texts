%%%%%%%%%%%%%%%%%%%%%%%%%%%%%%%%%%%%%%%%%
% Long Professional Curriculum Vitae
% LaTeX Template
% Version 1.1 (9/12/12)
%
% This template has been downloaded from:
% http://www.latextemplates.com
%
% Original author:
% Rensselaer Polytechnic Institute (http://www.rpi.edu/dept/arc/training/latex/resumes/)
%
% Important note:
% This template requires the res.cls file to be in the same directory as the
% .tex file. The res.cls file provides the resume style used for structuring the
% document.
%
%%%%%%%%%%%%%%%%%%%%%%%%%%%%%%%%%%%%%%%%%

%----------------------------------------------------------------------------------------
%	PACKAGES AND OTHER DOCUMENT CONFIGURATIONS
%----------------------------------------------------------------------------------------

\documentclass[12pt]{res} % Use the res.cls style, the font size can be changed to 11pt or 12pt here
\usepackage{helvet} % Default font is the helvetica postscript font
%\usepackage{newcent} % To change the default font to the new century schoolbook postscript font uncomment this line and comment the one above

\newsectionwidth{0pt} % Stops section indenting

\pagestyle{plain}
\begin{document}

%----------------------------------------------------------------------------------------
%	YOUR NAME AND ADDRESS(ES) SECTION
%----------------------------------------------------------------------------------------

\name{Agung Prakasya\\ \\} % Your name at the top

% If you don't want one of the addresses, simply remove all the text in the first or second \address{} bracket

\address{{\bf Contact Information} \\ E: agung.prakasya@outlook.co.id \\ M: +62 8.1975.1975.0} % Your address 1

%----------------------------------------------------------------------------------------

\begin{resume}

%----------------------------------------------------------------------------------------
%	OBJECTIVE SECTION
%----------------------------------------------------------------------------------------

\section{\centerline{Personal Info}}

\vspace{8pt} % Gap between title and text

{\sl Born}: Ciamis, 20 May 1975; Live in East of Cibubur, West Java; {\sl Nationality}: Indonesian; {\sl Language}: Bahasa Indonesia \& English\\ 

%----------------------------------------------------------------------------------------
%	EDUCATION SECTION
%----------------------------------------------------------------------------------------

\section{\centerline{EDUCATION}} 

\vspace{8pt} % Gap between title and text

{\sl Degree in Math}\\
Math Dept. of Bandung Institute of Technology, Bandung \hfill April 2000\\ 
Majoring in Math Modelling;

%----------------------------------------------------------------------------------------
 
\vspace{0.2in} % Some whitespace between sections

%----------------------------------------------------------------------------------------
%	COMPUTER SKILLS SECTION
%----------------------------------------------------------------------------------------

\section{\centerline{QUALIFICATION}}

\vspace{8pt} % Gap between title and text

Technology trends anticipating; Managing team of large size and/or high-skilled peoples; Mixing \& matching personnels to achieve effective team composition; 
Excellent polyglot coder; Long-time {\sl command-liner}; Very good on problem analysis and systematic solving, very good on statistical data analysis and math modelling.


%----------------------------------------------------------------------------------------

\vspace{0.2in} % Some whitespace between sections

%----------------------------------------------------------------------------------------
%	SUMMARY SECTION
%----------------------------------------------------------------------------------------

\section{\centerline{EXPERIENCE SUMMARY}} 

\vspace{8pt} % Gap between title and text
Managed large multi sites team of 150s persons, including developers, operators and managers for a 3 years managed service project; 
Managed a team of 3 exceptionally high skilled persons for a nationwide project;
Architected various systems: {\sl SCADA--enterprise integration}, {\sl Bigdata migration}, nationwide {\sl distributed biometric matching system}, 
{\sl datastreams}, {\sl Core Banking--stream offloading}.
{\sl 19 Years} of total experience; 5 years in {\sl Bigdata and Datastream}; 11 years of experience in {\sl energy industries}: Oil \& Gas and Mining ---mostly on
{\sl project management \& specialist position}; 

%----------------------------------------------------------------------------------------
 
\vspace{0.2in} % Some whitespace between sections

%----------------------------------------------------------------------------------------
%	PROFESSIONAL EXPERIENCE SECTION
%----------------------------------------------------------------------------------------

\section{\centerline{EXPERIENCE}} 

\vspace{8pt} % Gap between title and text
{\bf \uppercase{Sibernetik Integra Data}--Jakarta} \hfill {\sl Dec 2017--present} \\
Determines company's technology related strategies\hfill {\sl Head of Solution Dev.}\\including roadmap, stacks, team skill sets and partnerships.

\vspace{10pt} 
\begin{description}
    \item[{\sl Q4.2019.}] {\bf Information War Campaign--BSSN}\hfill {\sl Technology Advisor}\\
    To provide media monitoring tools to BSSN regarding issues in social and formal media. Functions: {\sl Sentiment analysis}, bot detection,
    social media monitoring, etc.
    \item[{\sl Q4.2019.}] {\bf Sistem Koordinasi Terpadu CAKRA--KOSTRAD}\hfill {\sl Technology Advisor}\\
    To provide media monitoring tools to KOSTRAD regarding issues in social and formal media, personnel distribution, war and peace strategies etc.
    Functions: {\sl Sentiment analysis}, social media monitoring, etc.
    \item[{\sl 2019--2020.}] {\bf Bigdata \& Datastreaming Integration\hfill {\sl Technology Advisor}\\--Commonwealth Bank}\\
    To provide {\sl Confluent Kafka} data streaming solution to Commonwealth Bank Digital Banking. Functions: Realtime notification, realtime data ingestion, etc.
    \item[{\sl 2019--2022.}] {\bf Bigdata \& Datastreaming Integration--BRI} \hfill {\sl Senior Solution Architect}\\
    To provide {\sl Confluent Kafka} data streaming solution to BRI Digital Banking. Functions: Realtime notification, realtime data ingestion, fraud detection,
    cash management, etc.
    \item[{\sl Q3.2019.}] {\bf AIS live data streaming to Bigdata integration\hfill {\sl Technology Advisor}\\--Bea Cukai}\\
    To provide {\sl Confluent Kafka} data streaming solution to Ditjen Beacukai. Functions: Realtime AIS data streaming from multiple sources, on the fly data query,
    on the fly data quality etc.
    \item[{\sl Q1.2019.}] {\bf Operational Dashboard--BAKAMLA} \hfill {\sl Technology Advisor}\\
    To provide dashboard solution to BAKAMLA for their strategic and day-to-day operation. Functions: Realtime notification, organizer, maps etc.
    \item[{\sl Q1.2019.}] {\bf Ceria Payment Gateway--BRI} \hfill {\sl Technology Advisor}\\
    To implement in-merchant-app payment page and BRI side gateway for BRI ceria microloan.
    \item[{\sl Q4.2018.}] {\bf Ceria Collection System--BRI} \hfill {\sl Technology Advisor}\\
    To design \& implement collection system for BRI Ceria, a microloan for online merchant's customers.
    \item[{\sl Q4.2018.}] {\bf Ceria Merchant Dashboard--BRI} \hfill {\sl Technology Advisor}\\
    To design \& implement dashboard for BRI merchant (at the time this was written {\sl Tokopedia \& Panorama})
    \item[{\sl 2017--2018.}] {\bf Distributed Biometric Matching System\hfill {\sl Solution Architect, PM, Dev}\\--Ditjen Imigrasi}\\
    To design \& implement distributed biometric matching service for other subsystems in SIMKIM v.2 of Ditjen Imigrasi such as Izin Tinggal, Visa, Layanan Paspor, Cekal etc.
\end{description}

{\bf \uppercase{Dattabot}--Jakarta} \hfill {\sl Aug 2017--Dec 2017} \\
To determine bigdata architecture and data migration methodology \hfill {\sl Consultant}

\vspace{10pt} 
\begin{description}
    \item[{\sl Q3--Q4.2017.}] {\bf Bigdata migration--Bank Danamon}\hfill {\sl Sr. Solution Architect}\\
    This project was to perform data migration from various data sources and core banking system to client's new data warehouse. This project
    employ various technologies such as {\sl Cloudera Hadoop}, {\sl Talend}, {\sl Sqoop} etc.
\end{description}

{\bf \uppercase{Dowedo}--Jakarta} \hfill {\sl Sep 2016--Jan 2018} \\
To design system architecture and implement Online Tourism\hfill {\sl Consultant}\\eCommerce platform 

\vspace{10pt} 
\begin{description}
    \item[{\sl 2016--2018.}] {\bf Building Tourism eCommerce Platform}\hfill {\sl Solution Architect}\\
    This project was to build tourism e-Commerce platform. However the life cycle was ended when it reached MVP.
\end{description}

{\bf \uppercase{TRIPATRA ENGINEERS \& CONSTRUCTORS}--Jakarta} \hfill {\sl Nov 2013--Apr 2015} \\[2pt]
To ensure that project support systems keep aligned with all project \hfill {\sl Sr. Engineer}\\
requirements despite characteristics difference between projects.

\vspace{10pt} 
\begin{description}
    \item[{\sl Q3.2014.}] {\bf Electronic Document Management System (EDMS)\hfill {\sl Project Manager}\\--Internal project}\\
    This internal project objective was to develop a new major release of the system. The new release
    was the result of knowledge absorption from on going projects to keep the system
    aligned with future projects requirements.
    \item[{\sl Q2.2014.}] {\bf Issue Management System--Internal project}\hfill {\sl Project Manager}\\
    This internal project objective was to develop a system to manage projects issues. The main purpose of the system is, to track issue statuses, 
    expedite for issue solutions and manage issue solution as lesson learned. 
    \item[{\sl Q1.2014.}] {\bf Project Control Portal--Internal project}\hfill {\sl Project Manager}\\
    This internal project objective was to develop a centralized information service for the corporate on all projects project control. 
    With this system, corporate are able to track progress status of the on going projects.
\end{description}
 
{\bf \uppercase{Sintesa Solusi Utama}--Java, Kalimantan, Sumatera} \hfill {\sl Feb 2006--Jun 2013} \\
It was a self-owned company which was founded as information technology service provider.

\vspace{10pt} 
\begin{description}
    \item[{\sl 2007--2013.}] {\bf Fuel Distribution Management System}\hfill {\sl Project Manager}\\({\sl Java, Sumatera})--{\bf Elnusa Petrofin}\\
    This project was a managed service which provide IT service for client's whole fuel distribution operation, including IT systems, infrastructures
    and human resources.
    \item[{\sl Q3.2007.}] {\bf Interactive 3D simulation of Sidoarjo Mud Burst\hfill {\sl Project Manager}\\--Lapindo Brantas}\\
    The project was to build interactive 3D simulation of the Sidoarjo mud on Banjar Panji site. This project used data provided by Lapindo Brantas 
    to simulate the chronology of Sidoarjo mud burst and was to be presented by client to government and parliament.
\end{description}

{\bf Professional Service: Multi-remote Siemens WinCC integration} \hfill {\sl Aug 2011}\\
{\sl Client}: {\bf Perusahaan Gas Negara}\hfill {\sl OPC Specialist}

\vspace{-10pt} 
The project was to integrate multi-stations Siemens Simatic WinCC which were distributed
in different remote locations, South Sumatera and West Java using OPC communication standard.

{\bf Professional Service: WinCC--TOPServer Integration}\hfill {\sl Feb 2007}\\
{\sl Client}: {\bf Yamatake Berca} \hfill {\sl OPC Specialist}

\vspace{-10pt} 
The project was to integrate Siemens Simatic WinCC and Yamatake TOPServer, using OPC
communication standard for a Building Automation project.

{\bf Professional Service: SCADA--Enterprise Integration}\hfill {\sl Feb--Sep 2006}\\
{\sl Client}: {\bf Kaltim Prima Coal} \hfill {\sl OPC Specialist}

\vspace{-10pt} 
The project was to integrate the existing Siemens Simatic WinCC and corporate information system 
(Mincom Minetrak --a mining management solution), which was currently in implementation phase. The
project was using OPC communication standard to stream data from the SCADA system. The
project was im plemented in a live production system which did not tolerate shutdown.

{\bf \uppercase{SIEMENS INDONESIA}--Jakarta, Balikpapan, Handil} \hfill {\sl May 2004--Feb 2006} \\
{\bf MRS-H251 Obsolete ESD and F\&G System Revamping} \hfill {\sl Control System Engineer}

\vspace{-10pt} 
The project was Obsolete ESD and F\&G System Revamping in a plant operated by \textit{Total E\&P
Indonesie} - East Kalimantan. The objective is to revamp shutdown system which was handled by
ALSPHA and implement Siemens technology for the function. The project was also to integrate the
system being implement to the existing system which was Honeywell DCS.\\
It was a project with a distinctive level of difficulties and risks, because it involved a live
production system and diverse technologies, such as OPC (\textit{OLE for Process Control}), redundancy
support, failsafe and others. End result of the individual work here was an OPC sub-system which
bridging communication between Siemens Simatic WinCC and Honeywell APPNode which conform
to both communication and safety standard.

{\bf \uppercase{SELF EMPLOYED}--Jakarta} \hfill {\sl 2000--Dec 2004} \\
Running self-owned internet cafe \& gaming center along with \hfill {\sl Business Owner}

\vspace{-10pt} 
various sideline: such as printing services, photo-editing, banner \& commercial design, billing system development, computer repair etc.

%----------------------------------------------------------------------------------------

\vspace{0.2in} % Some whitespace between sections

%----------------------------------------------------------------------------------------
%	INTERESTS SECTION
%----------------------------------------------------------------------------------------

\section{\centerline{INTERESTS}} 

\vspace{-5pt} % Reduce space between section title and contents

\begin{center}
Coding, guitar \& basses, woodworking, gardening
\end{center} 

%----------------------------------------------------------------------------------------
\end{resume} 

\end{document}